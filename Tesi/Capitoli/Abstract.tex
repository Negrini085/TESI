\newpage
\addcontentsline{toc}{chapter}{Sommario}
\begin{abstract}
L'obiettivo di questo lavoro di tesi è valutare le dimensioni e le eccentricità dei dischi proto-planetari in dipendenza dei parametri del sistema binario che li ospita. 
L'analisi da noi effettuata si concentra sul mass-ratio $q$, sull'eccentricità della binaria $e$ e sulla viscosità del materiale, determinata mediante il parametro adimensionale $\alpha$. 
La dinamica dei dischi d'accrescimento facenti parte di un sistema multiplo è fortemente influenzata dalla presenza dei compagni del corpo attorno al quale orbitano: abbiamo studiato numericamente questo fenomeno per mezzo di FARGO3D.
I risultati ottenuti evidenziano come le dimensioni dei dischi d'accrescimento diminuiscano all'aumentare dell'eccentricità del sistema: osserviamo che le estensioni maggiori ad $e$ fissata si presentano per quegli oggetti che orbitano attorno alle stelle costituenti la maggior frazione di massa del sistema binario.
L'approccio quantitativo per la determinazione dell'eccentricità del disco $e_{disco}$ implementato in questo lavoro di tesi consente di verificare come $e_{disco}$ scali con l'eccentricità del sistema binario ospitante.
Il confronto delle estensioni dei dischi simulati con i valori presenti in letteratura consente di verificare un buon accordo fra risultati numerici e previsioni teoriche, con scarti massimi del $20\%$.
Il lavoro di tesi presente propone nuovi sviluppi come un'estensione dell'analisi del fenomeno a nuove regioni dello spazio dei parametri.
\end{abstract}