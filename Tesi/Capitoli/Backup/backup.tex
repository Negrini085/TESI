è un tipico esempio di metodo alle differenze finite. Consideriamo le seguenti espansioni di Taylor:
\begin{equation}
\Vec{r}(t\,+\,\delta t)\,=\,\Vec{r}(t)\,+\,\Vec{v}(t)\delta t\,+\,\frac{1}{2}\Vec{a}(t)\delta t^2\,+\,\frac{1}{6}\dot{\Vec{a}}(t)\delta t^3\,+\,o(\delta t^4),
\label{eq:exp1_verlet}
\end{equation}
\begin{equation}
\Vec{r}(t\,-\,\delta t)\,=\,\Vec{r}(t)\,-\,\Vec{v}(t)\delta t\,+\,\frac{1}{2}\Vec{a}(t)\delta t^2\,-\,\frac{1}{6}\dot{\Vec{a}}(t)\delta t^3\,+\,o(\delta t^4),
\label{eq:exp2_verlet}
\end{equation}
Notiamo che è possibile determinare la posizione a $t\,+\,\delta t$ una volta note la configurazione attuale e quella allo step evolutivo precente, poiché
\begin{equation}
\Vec{r}(t\,+\,\delta t) \simeq 2 \Vec{r}(t)\,-\,\Vec{r}(t\,-\,\delta t)\,+\,\Vec{a}(t)\delta t^2\,+\,o(\delta t^4).
\label{eq:pos_succ}
\end{equation}
Il termine $o(\delta t^4)$ rappresenta l'errore di troncamento.


: in \cite{Pichardo2005} tutte le particelle test sono lanciate quando la binaria si trova a periastro . La velocità delle particelle test $v_{test}$ è perpendicolare alla congiungente fra i due corpi.
L'analisi computazionale viene interrotta se la particella si allontana troppo ($\sim\,10\,a$) dal corpo attorno al quale orbita oppure se va ad accrescere la massa della stella. 




I parametri che delimitano la regione d'applicazione del dumping sono:
\begin{list}{\textbf{-}}{\setlength{\itemsep}{0cm}}
    \item $D_z$, ossia la variabile \textit{Damping Zone} che determina gli intervalli radiali dove viene applicato lo smorzamento
    \item $\tau_D$, che è il tempo caratteristico di damping (in unità dell'inverso della frequenza locale)
\end{list}
I limiti delle due corone circolari dove vengono applicate le \textit{Wave Killing BC} sono determinati a partire da $D_z$ come
\begin{equation}
r_{inf}\,=\,r_{min} \cdot (D_z)^{2/3},
\label{eq:r_inf_damp}
\end{equation}
\begin{equation}
r_{sup}\,=\,r_{max} \cdot (D_z)^{-2/3}.
\label{eq:r_sup_damp}
\end{equation}
Dalle relazioni precedenti risulta evidente che se $D_z\,<\,1$, allora in nessuna parte della griglia simulativa verranno smorzate le perturbazioni di densità.
Ad ogni cella facente parte delle regioni 'di smorzamento' è associata una quantità $R_y$, il cui valore viene calcolato come:
\begin{equation}
\begin{cases}
R_y\,=\,(r_{cella}-r_{sup})/(r_{max}-r_{sup}) \qquad r_{cella}\,>\,r_{sup}\\
R_y\,=\,(r_{inf}-r_{cella})/(r_{inf}-r_{min}) \qquad r_{cella}\,<\,r_{inf}\
\end{cases}
\label{eq:rampy_damp}
\end{equation}
dove $r_{cella}$ è la posizione radiale media della cella presa in considerazione. 