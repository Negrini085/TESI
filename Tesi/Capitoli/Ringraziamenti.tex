\newpage
\begin{acknowledgements}
Vorrei sfruttare questa ultima parte della tesi per esprimere la mia gratitudine a tutte le persone che mi hanno supportato durante questo lavoro di tesi.\\

In primo luogo desidero ringraziare i miei supervisori Giovanni Rosotti ed Enrico Ragusa per la loro guida preziosa e per il loro sostegno costante: i loro consigli mi hanno fornito degli spunti preziosi necessari per il completamento di questo lavoro.\\

Desidero dedicare questo mio lavoro a tutti i miei amici con cui ho condiviso il percorso di studi e che hanno reso più divertenti i momenti passati in dipartimento. 
Voglio ringraziare di cuore Fragaini, che sebbene come sport preferito pratichi "Mettere in imbarazzo Filo in ogni possibile contesto sociale" è stato (ed è ancora adesso) un amico sincero, una spalla su cui piangere (perché siamo onesti potrebbe essere successo) ed una presenza positiva nella mia vita.
Ringrazio i membri di "Falce $\&$ Marcello" per la compagnia che mi hanno fatto in questi anni (e per aver partecipato come passeggeri alla mia prima driftata in Matiz) e tutta la Risotto per i momenti di cazzeggio intenso giocando a GeoGuessr. \\

Ringrazio OnlyMassy per i nostri piotti montagnini che sincero la gasano non poco: secondo me in futuro una puntatina sopra i 4000 è un obiettivo più che possibile da raggiungere. Parlando di giri in the wild non posso non menzionare Pino, che sebbene negli ultimi anni non veda di frequente come un tempo, è sempre stato un compagno affidabile con cui smaltire gli sbatti dell'uni (forse lanciando sassi dalle montagne o forse no, sarà il lettore a giudicare...).\\

Ricordo con piacere (oddio se non avessi dovuto studiare e non avessi avuto male alle ginocchia con più piacere) le sessioni estive passate in montagna con mio cugino Fabio: sappi che anche se non vado più in bici da corsa come un tempo, morirò sulla sella piuttosto che farmi staccare. Ci tengo a rimarcare che non sono competitivo :-).\\

Ringrazio la gang dell'arrampicata per le serate scialle che abbiamo passato a Manga e per avermi sopportato in sti mesi di sbatti. L'unico problema di questa cumpa è quel giovane stallone italiano con la barba vorticosa che arriva da "Ba Ba Barona" conosciuto come Lucio Latin Lover che oltre ad essere enorme, mi ruba tutte le tipe.\\

Ringrazio tutto il popolo della BICF per le pausette durante lo studio. Sincero le serate BICF la gasano non poco, dopo le 9 di sera in quel posto mi sento a casa.\\

Ringrazio i miei compagni delle superiori per i piazzi in Piazza Leo durante le sessioni e un po' durante tutto l'anno: necessaria una reunion per vedere come stanno i \textit{Raton}.\\

Un ringraziamento particolare va a Pierangela Lentini, la mia professoressa di fisica delle superiori: anche se mi "ha venduto un po' un accollo di corso di laurea", le sarò per sempre grato per l'impatto che la sua passione nell'insegnamento ha avuto su di me.\\

Infine ringrazio i miei genitori: solo grazie al loro sacrificio costante negli anni (anche prima che nascessi) ho avuto la possibilità di continuare a studiare. 
Sebbene gli anni di uni con il covid non sono stati facili e a volte abbiamo avuto delle incomprensioni, li ringrazio per il sostegno che mi hanno fornito in questi ultimi mesi e per l'empatia che hanno mostrato nei miei confronti. \

%Grazie a tutti coloro che hanno contribuito a rendere possibile questo traguardo. \footnote{scherzone questo l'ha scritto ChatGPT, vi voglio bene}
\end{acknowledgements}