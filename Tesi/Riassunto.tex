\documentclass[12pt, a4paper]{report}
\pdfminorversion=4
\usepackage{caption, subcaption}
\usepackage[utf8]{inputenc}
\usepackage[italian]{babel}
\usepackage[T1]{fontenc}
\usepackage{graphicx}
\usepackage[top=3cm, bottom=3cm]{geometry}
\usepackage{tabularx}
\usepackage{titlesec}
\usepackage{colortbl}
\usepackage{csquotes}
\usepackage{amsmath}
\usepackage{amssymb}
\usepackage{xcolor}
\usepackage{array}
\usepackage{float}

\newcommand{\apj}{\textit{The Astrophysical Journal}}
\newcommand{\mnras}{\textit{Monthly Notices of the Royal Astronomical Society}}
\newcommand{\aap}{\textit{Astronomy \& Astrophysics}}
\newcommand{\nar}{\textit{New Astronomy Reviews}}
\newcommand{\araa}{\textit{Annual Review of Astronomy and Astrophysics}}

\usepackage[backend=biber, style=authoryear, maxcitenames=2, maxbibnames=99]{biblatex}
\renewcommand*{\finalnamedelim}{%
\ifnumgreater{\value{liststop}}{2}{\finalandcomma}{}%
  \addspace\&\space}
\renewcommand*{\postnotedelim}{\addspace\mkbibparens}
\addbibresource{bibliografia.bib}

\newcolumntype{C}[1]{>{\centering\arraybackslash}p{#1}}

\renewcommand*{\nameyeardelim}{\addspace}
\renewcommand*{\postnotedelim}{\addcolon\space}
\renewcommand*{\finentrypunct}{)}
\renewcommand{\chaptername}{Capitolo}

\begin{document}

\begin{center}
    {\large\textbf{Studio del troncamento di dischi circumstellari: analisi della dipendenza dai parametri della binaria}}
\end{center}

L'obiettivo di questo lavoro di tesi è valutare le dimensioni e le eccentricità dei dischi proto-planetari circumstellari in dipendenza dei parametri del sistema binario che li ospita.
L'analisi da noi effettuata si concentra sul rapporto fra le masse della binaria $q$ (mass-ratio), sull'eccentricità della binaria $e$ e sulla viscosità del materiale.

I dischi proto-planetari sono delle strutture sottili costituite da gas e polveri che orbitano attorno ad una stella: in essi si formano i pianeti.
L'evoluzione del disco è dettata da meccanismi che determinano una ridistribuzione del momento angolare: il materiale che lo costituisce accresce lentamente sul corpo centrale.
Se un sistema stellare è composto da due stelle gravitazionalmente legate i dischi proto-stellari presenti possono essere tre: due circumstellari ed uno circum-binario.
Nell'ambito di questa tesi siamo interessati ai dischi circum-stellari, che abbiamo considerato essere posti sullo stesso piano orbitale della binaria.

Determinare l'estensione spaziale di un disco d'accrescimento in un sistema multiplo è una problematica che richiede lo studio dell'interazione fra un corpo perturbante ed un anello di gas.
Approcci semi-analitici come quello di \textcite{PapaloizouPringle1977}, focalizzato sull'analisi delle forze di marea causate dalla presenza del compagno, e quello di \textcite{GoldreichTremaine1980}, basato sulle risonanze fra le orbite del disco e della binaria, hanno consentito di determinare che la posizione in cui avviene il troncamento dipende da $e$ , da $q$, dal parametro adimensionale $\alpha$ che regola la viscosità del materiale e dalla temperatura $T$ del disco. 
Un campionamento fine dello spazio dei parametri è stato realizzato da \textcite{Pichardo2005} effettuando 45 simulazioni al variare di $q$ ed $e$: il metodo delle \textit{test particles} da loro utilizzato focalizza l'attenzione della ricerca su effetti puramente dinamici e non consente di studiare i fenomeni di natura viscosa che regolano l'evoluzione del disco.
Una trattazione più rigorosa del troncamento è quella di \textcite{ArtymowiczLubow1994}, che hanno preso in considerazione tutta la fisica del problema mediante simulazioni lagrangiane dell'evoluzione del disco: a causa dell'elevato costo computazionale i dischi circumstellari da loro studiati sono solamente otto.
\textcite{ManaraTronc2019} hanno proposto una formula analitica per la determinazione del raggio di troncamento
\begin{equation}
r_T\,=\,R_{L} (a e^b\,+\,c\mu^d),
\label{eq:tronc_disc}
\end{equation}
dove $R_L$ è la dimensione spaziale del Roche-Lobe \parencite{Eggleton1983}, $c\,=\,0.88$ e $d\,=\,0.01$ sono dei parametri determinati a partire dal lavoro di \textcite{PapaloizouPringle1977} ed $a$, $b$ sono stati calcolati fittando i risultati numerici ottenuti da \textcite{ArtymowiczLubow1994}.

Nonostante il troncamento sia stato diffusamente trattato dal punto di vista computazionale, una vasta regione dello spazio dei parametri resta ad oggi inesplorata: questo lavoro di tesi si propone di colmare alcune delle lacune presenti. 
Per fare ciò effettuiamo delle simulazioni numeriche 2D utilizzando FARGO3D \parencite{Fargo3D}, che è un codice a griglia euleriana sviluppato con l'obiettivo di studiare la fisica dei dischi d'accrescimento, ampiamente utilizzato in ambito astrofisico.
I sistemi binari che abbiamo analizzato possono essere divisi in tre categorie: circolari ($e\,=\,0.0$), a media eccentricità ($e\,=\,0.3$) e ad elevata eccentricità ($e\,=\,0.6$).
Per ogni valore di $e$, abbiamo studiato quattro rapporti fra la massa $m_2$ della stella secondaria e quella $m_1$ della primaria.
Abbiamo deciso di lavorare con tre differenti viscosità del materiale orbitante esplorando l'intervallo caratteristico del problema in analisi, giungendo a valori del numero di Reynolds più elevati rispetto a \textcite{ArtymowiczLubow1994}.
Il numero totale di simulazioni effettuate in questo lavoro di tesi è 72, nove delle quali sono delle run di convergenza per verificare la correttezza dell'approccio utilizzato: il campione da noi analizzato è 9 volte più grande di quello di \textcite{ArtymowiczLubow1994}.
Le dimensioni del disco e la sua eccentricità vengono valutate per ogni configurazione assunta dal materiale fra il trentesimo e il cinquantesimo periodo di rivoluzione della binaria: i risultati che forniamo sono i valori medi delle stime effettuate.

Abbiamo determinato l'estensione dei dischi con due metodi differenti: per ogni sistema in analisi effettuiamo una stima del raggio di troncamento e del semiasse maggiore del disco. 
Abbiamo osservato che ad $e$ fissato la dimensione del disco dipende fortemente dal valore di $m_{cen}/m_{per}$, dove $m_{cen}$ è la massa della stella al centro della griglia, mentre $m_{per}$ la massa del corpo perturbante: gli oggetti che orbitano attorno ad una frazione di massa della binaria più elevata sono più grandi.
All'aumentare di $e$ le dimensioni dei dischi circumstellari analizzati diminuiscono sensibilmente: questo andamento è dovuto al fatto che le risonanze eccentriche sono di intensità maggiore per un sistema ad alta $e$, con un conseguente spostamento della regione di troncamento \parencite{ArtymowiczLubow1994}.
\'E possibile osservare una dipendenza da $\alpha$: maggiore è la viscosità, maggiori sono le dimensioni del disco che otteniamo.

La seconda caratteristica a cui siamo interessati è $e_{disco}$, che abbiamo definito come l'eccentricità delle orbite percorse dal materiale nell'intorno del semiasse maggiore di troncamento.
Abbiamo effettuato una stima per ognuno dei 63 dischi simulati, osservando che in media $e_{disco}$ diminuisce all'aumentare dell'eccentricità della binaria: questo comportamento contro-intuitivo sembra si verifichi perché le risonanze responsabili dell'eccitazione di $e_{disco}$ finiscono per trovarsi a raggi maggiori rispetto a quello di troncamento.
\textcite{ArtymowiczLubow1994} fanno riferimento ad $e_{disco}$ solo per rimarcare come la mancanza di simmetria per rotazioni attorno all'asse $z$ del disco pregiudichi una corretta valutazione di $r_T$: l'approccio quantitativo presente in questa tesi è innovativo.

Abbiamo effettuato un confronto fra le dimensioni dei dischi da noi ottenute e quelle presenti in letteratura osservando un buon accordo fra i valori di $r_T$ per i dischi ospitati in sistemi binari circolari o altamente eccentrici. 
Nel caso di binaria ad $e\,=\,0.3$ abbiamo osservato delle discrepanze in media del $15\,\%$ fra i valori teorici e quelli da noi ottenuti: il modo di scalare delle dimensioni del disco è leggermente differente nei due casi.
Considerato il buon accordo fra i nuovi risultati numerici e quelli ottenuti da lavori precedenti, uno dei possibili sviluppi futuri di questo lavoro di tesi è un estensione del fit di $a,\,b,\,c,\,d$ ad una nuova regione dello spazio dei parametri.

\setlength{\bibitemsep}{\baselineskip}
\printbibliography[heading = bibintoc, title=Bibliografia]

\end{document}
